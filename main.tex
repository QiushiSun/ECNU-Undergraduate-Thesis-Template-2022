\documentclass{ecnuthesis}
% \documentclass[printMode]{ecnuthesis}
% 模版选项:
% printMode     是否开启打印模式, 若缺省则为关闭, 反之则为开启
% 用法示例
% \documentclass[printMode]{ecnuthesis}   (开启打印模式, 适合双面打印)
% \documentclass[printMode]{ecnuthesis}   (关闭打印模式, 适合提交电子版)


% \setCJKfamilyfont{simsun}{SimSun.ttf}
% \newcommand{\SimSun}{\CJKfamily{simsun}}

\ecnuSetup {
  % 参数设置
  % 允许采用两种方式设置选项:
  %   1. style/... = ...
  %   2. style = { ... = ... } 
  % 注意事项: 
  %   1. 请勿在参数设置中出现空行
  %   2. "=" 两侧的空格将被忽略
  %   3. "/" 两侧的空格不会被忽略
  %   4. 请使用英文逗号 "," 分隔选项
  %
  % info 类用于输入论文信息
  info = {
    title = {关于在数据科学与工程学院摸鱼的研究},
    % 中文标题
    %
    titleEN = {Research on Fish Touching in the School of Data Science and Engineering},
    % 英文标题
    %
    % 如果名字包含生僻字导致输出错误,请注释掉line34,取消注释line33,将名字输入在“包含生僻字的名字”,字体获取见README.md
    % author = {\CJKfontspec{SimSun.ttf}[AutoFakeBold = {3.17}]{包含生僻字的名字}},
    author = {钱院长的小迷弟},
    % 作者姓名
    %
    studentID = {10185501xxx},
    % 作者学号
    %
    department = {数据科学与工程学院},
    % 学院名称
    %
    major = {数据科学与大数据技术},
    % 专业名称
    %
    supervisor = {大橘猫},
    % 指导教师姓名
    %
    academicTitle = {副研究员},
    % 指导教师职称
    %
    year  = 2022,
    % 论文完成年份
    %
    month = 5,
    % 论文完成月份
    %
    keywords = {神经网络,摸鱼},
    % 中文关键词
    % 请使用英文逗号 "," 以分隔
    %
    keywordsEN = {Neural Networks, Fish Touching},
    % 英文关键词
    % 请使用英文逗号 "," 以分隔
    %
  },
  % style 类用于简单设置论文格式
  style = {
    footnote  = plain,
    % 脚注编号样式
    % 可用选项:
    %   footnote = plain|circled
    % 说明:
    %   plain     脚注的编号仅为数字
    %   circled   脚注的编号为带圆圈数字 (仅限1-10)
    %   (默认选项为 plain )
    %
    numbering = chinese,
    % 章节编号样式
    % 可用选项:
    %   numbering = arabic|alpha|chinese
    % 说明:
    %   arabic    使用数字进行编号 (即理科要求)
    %   alpha     使用字母进行编号 (即外文要求)
    %   chinese   使用汉字进行编号 (即文科要求)
    %   (默认选项为 arabic )
    %
    fontCJK = fandol,
    % 中文字体选择
    % 可用选项:
    %   fontCJK = fandol|windows|mac|default
    % 说明:
    %   fandol    使用 TeX 自带的 fandol 字体
    %   windows   使用 Windows 系统内的字体 (中易)
    %   mac       使用 MacOS 系统内的字体
    %   (默认选项为 fandol )
    %
    bibResource = {thesis-ref.bib},
    % 参考文献数据源
    % 由于使用的是 biber + biblatex , 所以必须明确给出 .bib 后缀名
    %
    logoResource = {./source/inner-cover(contains_font).eps},
    % 封面插图数据源
    % 模版已自带, 默认选项也已经设置为 ./source/inner-cover(contains_font).eps
  }
}



% 需要的宏包可以自行调用
\usepackage{mwe}

\begin{document}

% 设置前置部分编号
\frontmatter

% 中文摘要环境
\begin{abstract}
  
 这是摘要
\end{abstract}

% 英文摘要环境
\begin{abstractEN}
  
 This is my abstract
\end{abstractEN}

% 设置正文编号
\mainmatter

\chapter{绪论}
\section{背景与意义}
% \subsection{简介}
互联网的高速发展使得教育模式逐步信息化


\section{研究内容}


\section{组织结构}
本文共分为1024个章节,内容如下所示

% 新增:调整了itemize的间距
调整了itemize的间距
\begin{itemize}
    \item 第1部分为绪论部分。
    \item 第2部分主要对...
    \item 第3部分为模型设计与实现,是本研究工作的核心。该部分详述了实验设计、环境、参数选择以及其他细节。
    \item ...
\end{itemize}

\chapter{相关研究工作}

\section{来张图片试试}

% 新增:增加了双语Caption

新增了双语Caption

\begin{figure}[H]
    \centering
    \includegraphics[scale=0.2]{source/ecnu-dase.jpg}
    \bicaption{\centering 数据科学与工程学院}{School of Data Science and Engineering}
    % 如果中文/英文太长,可以使用 \centering 命令来控制其位置
\end{figure}


\section{来张表格试试}

\begin{table}[h]
\centering
\begin{tabular}{llll} 
  & A & B & C \\
  \hline complexity & $O(nd^{2})$ & $O({knd^{2}})$ & $O(n^{2}d)$ \\
  \hline path & $O(1)$ & $O(n)$ & $O(n)$ \\
\end{tabular}
  \bicaption{一张很可爱的表格}{A very cute tabular}
  \label{tab:my_label}
\end{table}

\section{来个引用试试}

BERT\cite{devlin-etal-2019-bert}是一种神奇的模型,本文的引用格式为gb7714-2015

\begin{figure}[H]
    \centering
    \includegraphics[scale=1.5]{source/bert_very_cute.png}
    \bicaption{\centering 芝麻街}{Sesame Street}
    % 如果中文/英文太长,可以使用 \centering 命令来控制其位置
\end{figure}

\chapter{模型设计与构建}

\section{参数设置}

\chapter{实验设计与分析}

\section{实验设置}

\section{数据来源}

% 正文后部分
\backmatter
% 导入参考文献 (需要通过 latexmk 编译后才能显示)
\printbibliography

% 新增:导入未被引用的参考文献 (需要通过 latexmk 编译后才能显示)
\nocite{zhang2021dive}
\nocite{Goodfellow-et-al-2016}
\nocite{qiu2020nndl}

% 附录环境
\begin{appendix}
 感谢数据科学与工程学院的老师对我的栽培。
\end{appendix}

% 致谢环境
\begin{acknowledgement}

感谢数据科学与工程学院\footnote{http://dase.ecnu.edu.cn/}的老师对我的栽培。
\end{acknowledgement}

\end{document}